\documentclass[
			12pt, %tamanho da fonte
			openright, % capítulos começam em página ímpar
			sumario=tradicional, %deixa o sumário "bonitinho"
			a4paper, %tamanho do papel
			twoside, %opção pra impressão frente e verso, caso queira só frente, use \oneside
			english, %idioma adicional
			brazil %idioma principal vem por último
			] 
			{abntex2} %classe abntex2 deixa nas normas e puxa uma série de funções úteis em ptbr

%PACOTES

% compilação de fontes

\usepackage{mathtools}
\usepackage{amsfonts}
\usepackage{amssymb}
\usepackage{amsmath}	
\usepackage[utf8]{inputenc}
%\usepackage[portuguese]{babel} desnecessário no abntex
\usepackage[T1]{fontenc}

%layouts
\usepackage{indentfirst} %cria parágrafo pra primeira linha do texto
\usepackage{multicol} %permite trabalhar em mais de uma coluna
\setlength{\columnsep}{1cm} %define separação entre as colunas

%outros pacotes úteis e específicos
\usepackage{caption} %outras opções de legenda
\usepackage{tikz} %pacote pra desenho
\usepackage[americanvoltages]{circuitikz} %desenhar circuitos
\usepackage{microtype} 				% para melhorias de justificação
\usepackage{icomma} %separador de decimal
\usepackage{units} %escreve unidades em romano dentro do modo matemático
\usepackage{notoccite} %impede que uma referência da lista de figuras seja contada como referência do trabalho, mudando a ordem com que elas são apresentadas
\usepackage{siunitx} %algum pacote pra unidades do SI acho que nem uso?
\usepackage{graphicx} %para inserir imagens
\usepackage{float} %permite escolher o local da imagem
\usepackage{lmodern} %latin modern
\usepackage{lipsum} % gerador de blablabla pra preencher texto
\usepackage[num]{abntex2cite} %pacote para citação do abntex em formato de número
\usepackage{subcaption} %permite criar sublegendas em figuras (a),(b),(c), etc.
\usepackage{tablefootnote} %permite criar uma nota de rodapé em tabela
\usepackage{titlesec} %customização de seções e capítulos
\usepackage{hyperref} %permite criar hyperlinks de autorreferências dentro do texto
\hypersetup{
	colorlinks=true, %links coloridos
	linkcolor=blue, %cor dos links
	citecolor=red % cor das citações
}
\usepackage{charter} %pacote para a fonte do texto

%%%%%%%%%%%%%%%%%%% MACROS %%%%%%%%%%%%%%%%%%%%%%%%%


\titulo{xx}
\autor{xx}
\data{xx}
\instituicao{xx}
\local{xx}
\tipotrabalho{xx}
\orientador{xx}
\preambulo{xx}

\counterwithin{figure}{chapter} %contar figuras por capítulo
\counterwithin{table}{chapter} %conta as tabelas por capítulo

\let\oldhat\hat 
\renewcommand{\vec}[1]{\mathbf{#1}} %fazendo vetores e vetores unitários com negrito e não com seta
\renewcommand{\hat}[1]{\oldhat{\mathbf{#1}}}

\renewcommand\thesubfigure{(\alph{subfigure})} 
\captionsetup[sub]{
	labelformat=simple
} %arruma a legenda da subfigura arrumar

\begin{document}
	\pretextual %elementos pré-textuais

		\begin{capa} % capa do trabalho
			\begin{center}	
				Ministério da Educação
				
				Secretaria de Educação Profissional e Tecnológica
				
				\imprimirinstituicao
				
				\textit{Campus} \imprimirlocal
				
				\vspace{1cm}
				
				\textbf{\imprimirautor}
				
				\vspace*{\fill}
				
				\textbf{\imprimirtitulo}
				
				\vfill
				\textbf{\imprimirlocal}
				
				\textbf{\imprimirdata}
				
				\thispagestyle{empty}
			\end{center}	
		\end{capa}

%%%%%%%%%%%%%%%%%%%%%%%%%%%%%%%%%%%%%%%%%%%%%%%%%%%%%%%%%%%%%%%%%%%%%%%%%%%%%%%%%%%%%%%%%%%%%%%%%%%%%%%%%%%%%%%%%%%%%%%%%%%%%%%%%%%%%%%%%%%%%%%%%%%%%%%%%%%%%%%%%%%%%%%%%%%%%%%%%%%%%%%%%%%%%%%%%%%%%%%%%%%%%%%%%%%%%%%%%%%%%%%%%%%%

\newpage

	\renewcommand{\folhaderostocontent}{ %folha de rosto redefinida
		\begin{center}
	
			Ministério da Educação
			
			Secretaria de Educação Profissional e Tecnológica
			
			\imprimirinstituicao
			
			\textit{Campus} \imprimirlocal
			
			\vspace{1cm}
			
			\textbf{\imprimirautor}
			
			\vspace*{\fill}	
			
			\textbf{\imprimirtitulo}
			
			\vspace*{\fill}
			
			\small
			
			\hspace{\fill}
			\begin{minipage}{.4 \linewidth}
				\imprimirpreambulo
				
				\vspace{.5cm}
				
				Professor orientador: \imprimirorientador
			\end{minipage}
		
			\vfill
			\normalsize
			\textbf{xx}
			
			\textbf{xx}
	
		\end{center}
	}
	
	\imprimirfolhaderosto %imprime a folha de rosto que foi definida
	
%%%%%%%%%%%%%%%%%%%%%%%%%%%%%%%%%%%%%%%%%%%%%%%%%%%%%%%%%%%%%%%%%%%%%%%%%%%%%%%%%%%%%%%%%%%%%%%%%%%%%%%%%%%%%%%%%%%%%%%%%%%%%%%%%%%%%%%%%%%%%%%%%%%%%%%%%%%%%%%%%%%%%%%%%%%%%%%%%%%%%%%%%%%%%%%%%%%%%%%%%%%%%%%%%%%%%%%%%%%%%%%%%%%%
	
	 % criando a folha de aprovacao
		\begin{center}
			\textbf{\imprimirautor}
			\vfill
			\textbf{\imprimirtitulo}
			\vfill
			\hspace{\fill}
			\begin{minipage}{.4 \linewidth}
				x
			\end{minipage}
			\vfill
			Rio do Sul, 26 de fevereiro de 2021
			\assinatura{Orientador \imprimirorientador \\ Instituto Federal Catarinense}
			\vfill
			\textbf{BANCA EXAMINADORA}
			\vfill
			\assinatura{x}
			\vfill
			\assinatura{x}
		\end{center}
	
%%%%%%%%%%%%%%%%%%%%%%%%%%%%%%%%%%%%%%%%%%%%%%%%%%%%%%%%%%%%%%%%%%%%%%%%%%%%%%%%%%%%%%%%%%%%%%%%%%%%%%%%%%%%%%%%%%%%%%%%%%%%%%%%%%%%%%%%%%%%%%%%%%%%%%%%%%%%%%%%%%%%%%%%%%%%%%%%%%%%%%%%%%%%%%%%%%%%%%%%%%%%%%%%%%%%%%%%%%%%%%%%%%%%	
	\newpage	
	\begin{epigrafe}
		\vspace*{\fill}
		\begin{flushright}
			\small
			\textquotedblleft x.\textquotedblright \  
			
			(\textit{x})
		\end{flushright}
	\end{epigrafe}
	\cleardoublepage %não deixa página em branco após
	
%%%%%%%%%%%%%%%%%%%%%%%%%%%%%%%%%%%%%%%%%%%%%%%%%%%%%%%%%%%%%%%%%%%%%%%%%%%%%%%%%%%%%%%%%%%%%%%%%%%%%%%%%%%%%%%%%%%%%%%%%%%%%%%%%%%%%%%%%%%%%%%%%%%%%%%%%%%%%%%%%%%%%%%%%%%%%%%%%%%%%%%%%%%%%%%%%%%%%%%%%%%%%%%%%%%%%%%%%%%%%%%%%%%%	
	\renewcommand{\resumoname}{{\fontfamily{bch}\selectfont Resumo}} %deixa o título do resumo na mesma fontfamily do resto do trabalho
	\begin{resumo}
		\lipsum[1-2]
		\vspace{\onelineskip}
		\noindent
		\textbf{Palavras-chave}: x, y, z
	\end{resumo}

	\begin{resumo}[{\fontfamily{bch}\selectfont Abstract}] %deixa o título na mesma fontfamily do resto do trabalho
		\selectlanguage{english} %puxa a lingua inglesa pro trabalho
			\lipsum[1-2]
			\vspace{\onelineskip}
			\noindent
			\textbf{Keywords}: x, y, z
	\end{resumo}

	\selectlanguage{brazil} %volta ao português

%%%%%%%%%%%%%%%%%%%%%%%%%%%%%%%%%%%%%%%%%%%%%%%%%%%%%%%%%%%%%%%%%%%%%%%%%%%%%%%%%%%%%%%%%%%%%%%%%%%%%%%%%%%%%%%%%%%%%%%%%%%%%%%%%%%%%%%%%%%%%%%%%%%%%%%%%%%%%%%%%%%%%%%%%%%%%%%%%%%%%%%%%%%%%%%%%%%%%%%%%%%%%%%%%%%%%%%%%%%%%%%%%%%%

	\renewcommand\listfigurename{{\fontfamily{bch}\selectfont Lista de Figuras}} %renomeia e deixa o título da lista de figuras na mesma fontfamily do trabalho
	\pdfbookmark[0]{\listfigurename}{lof}
	\listoffigures*
	\cleardoublepage
	
%%%%%%%%%%%%%%%%%%%%%%%%%%%%%%%%%%%%%%%%%%%%%%%%%%%%%%%%%%%%%%%%%%%%%%%%%%%%%%%%%%%%%%%%%%%%%%%%%%%%%%%%%%%%%%%%%%%%%%%%%%%%%%%%%%%%%%%%%%%%%%%%%%%%%%%%%%%%%%%%%%%%%%%%%%%%%%%%%%%%%%%%%%%%%%%%%%%%%%%%%%%%%%%%%%%%%%%%%%%%%%%%%%%%	
	
	\renewcommand\contentsname{{\fontfamily{bch}\selectfont Sumário}} %arruma o título do sumário e deixa na mesma fonte do resto do trabalho
	\tableofcontents
	
%%%%%%%%%%%%%%%%%%%%%%%%%%%%%%%%%%%%%%%%%%%%%%%%%%%%%%%%%%%%%%%%%%%%%%%%%%%%%%%%%%%%%%%%%%%%%%%%%%%%%%%%%%%%%%%%%%%%%%%%%%%%%%%%%%%%%%%%%%%%%%%%%%%%%%%%%%%%%%%%%%%%%%%%%%%%%%%%%%%%%%%%%%%%%%%%%%%%%%%%%%%%%%%%%%%%%%%%%%%%%%%%%%%%
	
	\textual %aqui começa o texto
	
	\renewcommand{\ABNTEXchapterfont}{\fontfamily{bch}\selectfont} %deixa a fonte dos capítulos na mesma fonte do trabalho
	\renewcommand{\ABNTEXchapterfontsize}{\HUGE}  %modifica o tamanho e o tipo da fonte para os capítulos


	\chapter*[Introdução]{Introdução}

	\lipsum[1-2]
	
	\chapter{Título 1}

	\lipsum[3-4]

	\end{document}